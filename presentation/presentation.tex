\documentclass{beamer}
\usetheme{Warsaw}
\hypersetup{pdfpagemode=FullScreen}

\usepackage[utf8]{inputenc}
\usepackage{default}
\usepackage[italian]{babel}
\usepackage{amsmath}
\usepackage{amsfonts}

\usepackage{graphicx}

\newcommand{\der}[2]{\frac{\partial #1}{\partial #2}}
\newcommand{\dder}[2]{\frac{\partial^2 #1}{\partial #2^2}}
\newcommand{\dmix}[3]{\frac{\partial^2 #1}{\partial #2 \partial #3}}

\AtBeginSection[]
{
  \begin{frame}
    \frametitle{Table of Contents}
    \tableofcontents[currentsection]
  \end{frame}
}

\begin{document}
%%%%%%%%%%%%%%%%%%%%%%%%%%%%%%%%%%%%%%%%%%%%%%%%%%%%%%%%%%%%%%%%%%%%%%%%%%%%%%%%%%%%%%%%%%%%%%%%%%%%%%%%%%%%%%%%%%%%%%%%%%%%%%%%%%%%%%%%%%%%%%%%%%%%%%%%%%%%%%%%%%%%%%%%%%%%%%%%%%%%%%%%%%%%%%%%%%%%%%%%%%%%%%%%%%%%%%%%%%%%%%%%%%%%%%%%%%%%%%%%%%%%%%%%%%%%%%%%%%
\section{Introduzione}
\begin{frame}
\frametitle{Scopo del progetto}
\begin{block}{}
Lo scopo è di creare una piccola libreria per il pricing di derivati finanziari con il metodo degli elementi finiti, appoggiandosi sulla libreria \textsf{deal.ii}.
\end{block}
\begin{block}{Motivazioni}
La procedura pi\`u diffusa in finananza è di usare le differenze finite. Gli elementi finiti, seppure leggermente pi\`u complicati da implementare, presentano solo vantaggi.
\end{block}
\end{frame}

%%%%%%%%%%%%%%%%%%%%%%%%%%%%%%%%%%%%%%%%%%%%%%%%%%%%%%%%%%%%%%%%%%%%%%%%%%%%%%%%%%%%%%%%%%%%%%%%%%%%%%%%%%%%%%%%%%%%%%%%%%%%%%%%%%%%%%%%%%%%%%%%%%%%%%%%%%%%%%%%%%%%%%%%%%%%%%%%%%%%%%%%%%%%%%%%%%%%%%%%%%%%%%%%%%%%%%%%%%%%%%%%%%%%%%%%%%%%%%%%%%%%%%%%%%%%%%%%%%
\section{Il problema}

\begin{frame}
\frametitle{L'equazione da risolvere}
\only<-5>{\begin{multline*}
\alert<2>{\der{C}{t}+\frac{\sigma^2}{2}S^2\dder{C}{S}+rS\der{C}{S}-rC}+\\ + 
\alert<3>{\int_\mathbb{R}\left(C(t,Se^y)\alert<4>{-C(t,S)-S(e^y-1)\der{C}{S}(t,S)}\right)\nu(dy)}=0
\end{multline*}}

\only<6>{\begin{multline*}
\der{u}{t}+\left(r-\frac{\sigma^2}{2}\right)\der{u}{x}+\frac{\sigma^2}{2}\dder{u}{x}-ru\\
+\int_\mathbb{R}\left( u(t,x+y)-u(t,x)-(e^y-1)\der{u}{x}\right)\nu(dy)=0
\end{multline*}}
Con opportune condizioni al contorno.\\
\pause
Possiamo scomporre il problema in due parti
\begin{itemize}[<+->]
\item La parte differenziale, trattata in modo usuale con l'aiuto della libreria \textsf{deal.ii}
\item La parte integrale, trattata in modo esplicito ad ogni passaggio. \only<4>{Separabile in due pezzi.}
\end{itemize}
\pause
Trasformazioni \alert<5>{\emph{price}} e \alert<6>{\emph{logprice}}
\end{frame}

%%%%%%%%%%%%%%%%%%%%%%%%%%%%%%%%%%%%%%%%%%%%%%%%%%%%%%%%%%%%%%%%%%%%%%%%%%%%%%%%%%%%%%%%%%%%%%%%%%%%%%%%%%%%%%%%%%%%%%%%%%%%%%%%
\begin{frame}
 \frametitle{Scomposizione della parte integrale}
\only<1>{
Definendo in modo seguente le quantità
 \begin{align*}
 \hat{\alpha}&=\int_\mathbb{R}(e^y-1)\nu(y)dy \\
 \hat{\lambda}&=\int_\mathbb{R}\nu(y)dy
\end{align*}
l'equazione diventa
\begin{equation*}
\der{C}{t}+\frac{\sigma^2}{2}S^2\dder{C}{S}+(r-\hat{\alpha})S\der{C}{S}-(r+\hat{\lambda})C+ \int_\mathbb{R}C(t,Se^y)\nu(y)dy=0
\end{equation*}
}
\only<2>{
Analogamente per la trasformazione logprice si ha
\begin{align*}
\hat{\lambda}&=\int_{\mathbb{R}}\nu(y)dy,\\
\hat{\alpha}&=\int_{\mathbb{R}}(e^y-1)\nu(y)dy,
\end{align*}
con rispettiva equazione
\begin{equation*}
\der{u}{t}+\left(r-\frac{\sigma^2}{2}-\hat{\alpha}\right)\der{u}{x}+\frac{\sigma^2}{2}\dder{u}{x}-(r+\hat{\lambda})u+\int_\mathbb{R}u(t,x+y)\nu(y)dy=0
\end{equation*}
}
\end{frame}


%%%%%%%%%%%%%%%%%%%%%%%%%%%%%%%%%%%%%%%%%%%%%%%%%%%%%%%%%%%%%%%%%%%%%%%%%%%%%%%%%%%%%%%%%%%%%%%%%%%%%%%%%%%%%%%%%%%%%%%%%%%%%%%%
\begin{frame}
\frametitle{In due dimensioni}
\only<1>{
Con la trasformazione \emph{Price}
\begin{multline*}
 \der{C}{t}+(r-\hat{\alpha}_1)S_1\der{C}{S_1}+(r-\hat{\alpha}_2)S_2\der{C}{S_2}+\frac{\sigma^2_1}{2}S_1^2\dder{C}{S_1}+\frac{\sigma^2_2}{2}S_2^2\dder{C}{S_2}\\
 +\rho\sigma_1\sigma_2S_1S_2\dmix{C}{S_1}{S_2}-(r+\lambda_1+\lambda_2)C\\
 +\int_\mathbb{R}C(t,S_1e^y,S_2)\nu_1(y)dy+\int_\mathbb{R}C(t,S_1,S_2e^y)\nu_2(y)dy=0
\end{multline*}
}

\only<2>{
Con la trasformazione \emph {logprice}
\begin{multline*}
\der{u}{t}+\frac{\sigma_1^2}{2}\dder{u}{x_1}+\frac{\sigma_2^2}{2}\dder{u}{x_2}+\rho\sigma_1\sigma_2\dmix{u}{x_1}{x_2}+
\left(r-\frac{\sigma_1^2}{2}-\hat{\alpha_1}\right)\der{u}{x_1}\\+
\left(r-\frac{\sigma_2^2}{2}-\hat{\alpha_2}\right)\der{u}{x_2}-(r+\hat{\lambda_1}+\hat{\lambda_2})u\\+
\int_\mathbb{R}u(t,x_1+y,x_2)\nu_1(y)dy+
\int_\mathbb{R}u(t,x_1,x_2+y)\nu_2(y)dy=0
\end{multline*}
}

\end{frame}

%%%%%%%%%%%%%%%%%%%%%%%%%%%%%%%%%%%%%%%%%%%%%%%%%%%%%%%%%%%%%%%%%%%%%%%%%%%%%%%%%%%%%%%%%%%%%%%%%%%%%%%%%%%%%%%%%%%%%%%%%%%%%%%%




%%%%%%%%%%%%%%%%%%%%%%%%%%%%%%%%%%%%%%%%%%%%%%%%%%%%%%%%%%%%%%%%%%%%%%%%%%%%%%%%%%%%%%%%%%%%%%%%%%%%%%%%%%%%%%%%%%%%%%%%%%%%%%%%%%%%%%%%%%%%%%%%%%%%%%%%%%%%%%%%%%%%%%%%%%%%%%%%%%%%%%%%%%%%%%%%%%%%%%%%%%%%%%%%%%%%%%%%%%%%%%%%%%%%%%%%%%%%%%%%%%%%%%%%%%%%%%%%
\section{Struttura del codice}

\begin{frame}
\frametitle{La libreria \textsf{deal.ii}}
\begin{block}{Libreria \textsf{deal.ii}}
Una potente libreria ad elementi finiti sui quadrilateri. Molto completa e semplice da iniziare a utilizzare, permette di risolvere problemi variazionali fino a 3 dimensioni con poche righe di codice. 
\end{block}

\pause
\begin{block}{Vantaggi}
 \begin{itemize}
  \item Documentazione molto ampia e chiara, a cui si aggiunge la presenza di 51 tutorial programs che illustrano come usare la libreria per problemi tipici
  \item Organizzata in moduli che coprono le diverse aree di un problema ad elementi finiti (\emph{creazione griglie, algebra lineare, output risultati, etc})
 \end{itemize}
\end{block}
\end{frame}
%%%%%%%%%%%%%%%%%%%%%%%%%%%%%%%%%%%%%%%%%%%%%%%%%%%%%%%%%%%%%%%%%%%%%%%%%%%%%%%%%%%%%%%%%%%%%%%%%%%%%%%%%%%%%%%%%%%%%%%%%%%%%%%%%

\begin{frame}
 \frametitle{La nostra implementazione }
 \begin{block}{Classi Opzione}
 Seguendo la linea di \textsf{deal.ii}, è stata creata una serie classi principali che rappresentano il problema e gestiscono creazione griglia, assemblaggio sistema e soluzione.
 \end{block}
 \begin{block}{Classi Model}
 I vari modelli utilizzati in finanza sono 
 \end{block}

\end{frame}


%%%%%%%%%%%%%%%%%%%%%%%%%%%%%%%%%%%%%%%%%%%%%%%%%%%%%%%%%%%%%%%%%%%%%%%%%%%%%%%%%%%%%%%%%%%%%%%%%%%%%%%%%%%%%%%%%%%%%%%%%%%%%%%%%%%%%%%%%%%%%%%%%%%%%%%%%%%%%%%%%%%%%%%%%%%%%%%%%%%%%%%%%%%%%%%%%%%%%%%%%%%%%%%%%%%%%%%%%%%%%%%%%%%%%%%%%%%%%%%%%%%%%%%%%%%%%%%%%%
\section{Risultati}

%%%%%%%%%%%%%%%%%%%%%%%%%%%%%%%%%%%%%%%%%%%%%%%%%%%%%%%%%%%%%%%%%%%%%%%%%%%%%%%%%%%%%%%%%%%%%%%%%%%%%%%%%%%%%%%%%%%%%%%%%%%%%%%%%%%%%%%%%%%%%%%%%%%%%%%%%%%%%%%%%%%%%%%%%%%%%%%%%%%%%%%%%%%%%%%%%%%%%%%%%%%%%%%%%%%%%%%%%%%%%%%%%%%%%%%%%%%%%%%%%%%%%%%%%%%%%%%%%%
\section{Conclusioni}

\end{document}