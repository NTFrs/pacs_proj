\documentclass{beamer}
\usetheme{Warsaw}
\hypersetup{pdfpagemode=FullScreen}

\usepackage[utf8]{inputenc}
\usepackage{default}
\usepackage[italian]{babel}
\usepackage{amsmath}
\usepackage{amsfonts}

\usepackage{graphicx}

\newcommand{\der}[2]{\frac{\partial #1}{\partial #2}}
\newcommand{\dder}[2]{\frac{\partial^2 #1}{\partial #2^2}}
\newcommand{\dmix}[3]{\frac{\partial^2 #1}{\partial #2 \partial #3}}

\AtBeginSection[]
{
  \begin{frame}
    \frametitle{Table of Contents}
    \tableofcontents[currentsection]
  \end{frame}
}

\begin{document}
%%%%%%%%%%%%%%%%%%%%%%%%%%%%%%%%%%%%%%%%%%%%%%%%%%%%%%%%%%%%%%%%%%%%%%%%%%%%%%%%%%%%%%%%%%%%%%%%%%%%%%%%%%%%%%%%%%%%%%%%%%%%%%%%%%%%%%%%%%%%%%%%%%%%%%%%%%%%%%%%%%%%%%%%%%%%%%%%%%%%%%%%%%%%%%%%%%%%%%%%%%%%%%%%%%%%%%%%%%%%%%%%%%%%%%%%%%%%%%%%%%%%%%%%%%%%%%%%%%
\section{Introduzione}
\begin{frame}
\frametitle{Scopo del progetto}
\begin{block}{}
Lo scopo è di creare una piccola libreria per il pricing di derivati finanziari con il metodo degli elementi finiti, appoggiandosi sulla libreria \textsf{deal.ii}.
\end{block}
\begin{block}{Motivazioni}
La procedura pi\`u diffusa in finananza è di usare le differenze finite. Gli elementi finiti, seppure leggermente pi\`u complicati da implementare, presentano solo vantaggi.
\end{block}
\end{frame}

%%%%%%%%%%%%%%%%%%%%%%%%%%%%%%%%%%%%%%%%%%%%%%%%%%%%%%%%%%%%%%%%%%%%%%%%%%%%%%%%%%%%%%%%%%%%%%%%%%%%%%%%%%%%%%%%%%%%%%%%%%%%%%%%%%%%%%%%%%%%%%%%%%%%%%%%%%%%%%%%%%%%%%%%%%%%%%%%%%%%%%%%%%%%%%%%%%%%%%%%%%%%%%%%%%%%%%%%%%%%%%%%%%%%%%%%%%%%%%%%%%%%%%%%%%%%%%%%%%
\section{Il problema}

\begin{frame}
\frametitle{L'equazione da risolvere}
\only<-5>{\begin{multline*}
\alert<2>{\der{C}{t}+\frac{\sigma^2}{2}S^2\dder{C}{S}+r\der{C}{S}-rC}+\\ + 
\alert<3>{\int_\mathbb{R}\left(C(t,Se^y)\alert<4>{-C(t,S)-S(e^y-1)\der{C}{S}(t,S)}\right)\nu(dy)}=0.
\end{multline*}}

\only<6>{\begin{multline*}
\der{u}{t}+\left(r-\frac{\sigma^2}{2}\right)\der{u}{x}+\frac{\sigma^2}{2}\dder{u}{x}-ru\\
+\int_\mathbb{R}\left( u(t,x+y)-u(t,x)-(e^y-1)\der{u}{x}\right)\nu(dy)=0
\end{multline*}}

\pause
Possiamo scomporre il problema in due pezzi
\begin{itemize}[<+->]
\item La parte differenziale, trattata in modo usuale con l'aiuto della libreria \textsf{deal.ii}
\item La parte integrale, trattata in modo esplicito ad ogni passaggio. Separabile in due pezzi.
\end{itemize}
\pause
Trasformazioni \alert<5>{\emph{price}} e \alert<6>{\emph{logprice}}
\end{frame}
%%%%%%%%%%%%%%%%%%%%%%%%%%%%%%%%%%%%%%%%%%%%%%%%%%%%%%%%%%%%%%%%%%%%%%%%%%%%%%%%%%%%%%%%%%%%%%%%%%%%%%%%%%%%%%%%%%%%%%%%%%%%%%%%%
\begin{frame}

\end{frame}

%%%%%%%%%%%%%%%%%%%%%%%%%%%%%%%%%%%%%%%%%%%%%%%%%%%%%%%%%%%%%%%%%%%%%%%%%%%%%%%%%%%%%%%%%%%%%%%%%%%%%%%%%%%%%%%%%%%%%%%%%%%%%%%%%

\begin{frame}
\frametitle{La libreria \textsf{deal.ii}}

\end{frame}


%%%%%%%%%%%%%%%%%%%%%%%%%%%%%%%%%%%%%%%%%%%%%%%%%%%%%%%%%%%%%%%%%%%%%%%%%%%%%%%%%%%%%%%%%%%%%%%%%%%%%%%%%%%%%%%%%%%%%%%%%%%%%%%%%%%%%%%%%%%%%%%%%%%%%%%%%%%%%%%%%%%%%%%%%%%%%%%%%%%%%%%%%%%%%%%%%%%%%%%%%%%%%%%%%%%%%%%%%%%%%%%%%%%%%%%%%%%%%%%%%%%%%%%%%%%%%%%%%%
\section{Struttura del codice}

%%%%%%%%%%%%%%%%%%%%%%%%%%%%%%%%%%%%%%%%%%%%%%%%%%%%%%%%%%%%%%%%%%%%%%%%%%%%%%%%%%%%%%%%%%%%%%%%%%%%%%%%%%%%%%%%%%%%%%%%%%%%%%%%%%%%%%%%%%%%%%%%%%%%%%%%%%%%%%%%%%%%%%%%%%%%%%%%%%%%%%%%%%%%%%%%%%%%%%%%%%%%%%%%%%%%%%%%%%%%%%%%%%%%%%%%%%%%%%%%%%%%%%%%%%%%%%%%%%
\section{Risultati}

%%%%%%%%%%%%%%%%%%%%%%%%%%%%%%%%%%%%%%%%%%%%%%%%%%%%%%%%%%%%%%%%%%%%%%%%%%%%%%%%%%%%%%%%%%%%%%%%%%%%%%%%%%%%%%%%%%%%%%%%%%%%%%%%%%%%%%%%%%%%%%%%%%%%%%%%%%%%%%%%%%%%%%%%%%%%%%%%%%%%%%%%%%%%%%%%%%%%%%%%%%%%%%%%%%%%%%%%%%%%%%%%%%%%%%%%%%%%%%%%%%%%%%%%%%%%%%%%%%
\section{Conclusioni}

\end{document}