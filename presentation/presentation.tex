\documentclass{beamer}
\usetheme{Warsaw}

\usepackage[utf8]{inputenc}
\usepackage{default}
\usepackage[italian]{babel}
\usepackage{amsmath}
\usepackage{amsfonts}

\usepackage{graphicx}

\newcommand{\der}[2]{\frac{\partial #1}{\partial #2}}
\newcommand{\dder}[2]{\frac{\partial^2 #1}{\partial #2^2}}
\newcommand{\dmix}[3]{\frac{\partial^2 #1}{\partial #2 \partial #3}}

\AtBeginSection[]
{
  \begin{frame}
    \frametitle{Table of Contents}
    \tableofcontents[currentsection]
  \end{frame}
}

\begin{document}
\section{Introduzione}
\begin{frame}
\end{frame}

%%%%%%%%%%%%%%%%%%%%%%%%%%%%%%%%%%%%%%%%%%%%%%%%%%%%%%%%%%%%%%%%%%%%%%%%%%%%%%%%%%%%%%%%%%%%%%%%%%%%%%%%%%%%%%%%%%%%%%%%%%%%%%%%%%%%%%%%%%%%%%%%%%%%%%%%%%
\section{Il problema}

\begin{frame}
\frametitle{L'equazione da risolvere}
\begin{multline*}
\alert<2>{\der{C}{t}+\frac{\sigma^2}{2}S^2\dder{C}{S}+r\der{C}{S}-rC}+\\ + 
\alert<3>{\int_\mathbb{R}\left(C(t,Se^y)-C(t,S)-S(e^y-1)\der{C}{S}(t,S)\right)\nu(dy)}=0.
\end{multline*}
\pause
Possiamo scomporre il problema in due pezzi
\begin{itemize}[<+->]
\item La parte differenziale, trattata in modo usuale con l'aiuto della libreria \textsf{deal.ii}
\item La parte integrale, trattata in modo esplicito ad ogni passaggio
\end{itemize}
\end{frame}
%%%%%%%%%%%%%%%%%%%%%%%%%%%%%%%%%%%%%%%%%%%%%%%%%%%%%%%%%%%%%%%%%%%%%%%%%%%%%%%%%%%%%%%%%%%%%%%%%%%%%%%%%%%%%%%%%%%%%%%%%%%%%%%%%%%%%%%%%%%%%%%%%%%%%%%%%%
\section{Struttura del codice}

%%%%%%%%%%%%%%%%%%%%%%%%%%%%%%%%%%%%%%%%%%%%%%%%%%%%%%%%%%%%%%%%%%%%%%%%%%%%%%%%%%%%%%%%%%%%%%%%%%%%%%%%%%%%%%%%%%%%%%%%%%%%%%%%%%%%%%%%%%%%%%%%%%%%%%%%%%
\section{Risultati}

%%%%%%%%%%%%%%%%%%%%%%%%%%%%%%%%%%%%%%%%%%%%%%%%%%%%%%%%%%%%%%%%%%%%%%%%%%%%%%%%%%%%%%%%%%%%%%%%%%%%%%%%%%%%%%%%%%%%%%%%%%%%%%%%%%%%%%%%%%%%%%%%%%%%%%%%%%
\section{Conclusioni}

\end{document}
