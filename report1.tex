\documentclass[a4paper,10pt]{article}

\usepackage[utf8]{inputenc}
\usepackage[italian]{babel}
\usepackage{amsmath}
\usepackage{amsfonts}

%opening
\title{Riassunto Progetto Pacs}
\author{Nahuel Foresta, Giorgio Re}

\newcommand{\der}[2]{\frac{\delta #1}{\delta #2}}
\newcommand{\dder}[2]{\frac{\delta^2 #1}{\delta #2^2}}
\newcommand{\dmix}[3]{\frac{\delta^2 #1}{\delta #2 \delta #3}}
\begin{document}

\maketitle

\section{Riassunto Obiettivi}

L'obiettivo del progetto è di creare un programma per prezzare una serie di opzioni utilizzando dei metodi basati a elementi finiti. Il valore di un opzione al variare del sottostante (che supponiamo evolvere secondo un modello Jump-Diffusion) può essere quasi sempre trovato come soluzione di un equazione integro differenziale del tipo

\begin{multline}
 \der{C}{t}+\frac{\sigma^2}{2}S^2\dder{C}{S}+r\der{C}{S}-rC+\\+ \int_\mathbb{R}\left(C(t,Se^y)-C(t,S)-S(e^y-1)\der{C}{S}(t,S)\right)k(y)dy=0
\end{multline}

su $[0,T]\times[0,+\infty]$ con opportune condizioni al bordo e condizione finale $C(T,S)=g(S)$ payoff dell'opzione. $k$ è un nucleo con una forte massa nell'intorno dello zero e code esponenziali.
In due dimensioni tale equazione diventa:


\begin{multline}
 \der{C}{t}+\frac{\sigma_1^2}{2}S_1^2\dder{C}{S_1}+\frac{\sigma_2^2}{2}S_2^2\dder{C}{S_2}+\rho\sigma_1\sigma_2 S_1 S_2 \dmix{C}{S_1}{S_2}+
 r\der{C}{S_1}+r\der{C}{S_2}-rC+ \\
 + \int_\mathbb{R^2}\left(C(t,S_1e^{y_1},S_2e^{y_2})-C(t,S_1,S_2)-S(e^y-1)\der{C}{S}(t,S)\right)k(y)dy=0 
\end{multline}
su $[0,T]\times[0,+\infty]^2$ con opportune B.C. e valore finale.


\section{}

\end{document}
