\documentclass[a4paper,10pt]{article}

\usepackage[utf8]{inputenc}
\usepackage[italian]{babel}
\usepackage{amsmath}
\usepackage{amsfonts}
\usepackage{xfrac}
%opening
\title{Riassunto Iniziale Progetto Pacs}
\author{Nahuel Foresta, Giorgio Re}

\setlength{\parindent}{0pt}

\newcommand{\der}[2]{\frac{\delta #1}{\delta #2}}
\newcommand{\dder}[2]{\frac{\delta^2 #1}{\delta #2^2}}
\newcommand{\dmix}[3]{\frac{\delta^2 #1}{\delta #2 \delta #3}}
\begin{document}

\maketitle

\section{Riassunto Obiettivi}

L'obiettivo del progetto è di creare un programma per prezzare una serie di opzioni utilizzando dei metodi basati a elementi finiti. Il valore di un opzione al variare del sottostante (che supponiamo evolvere secondo un modello Jump-Diffusion) può essere quasi sempre trovato come soluzione di un equazione integro differenziale del tipo

\begin{multline}
 \der{C}{t}+\frac{\sigma^2}{2}S^2\dder{C}{S}+r\der{C}{S}-rC+\\+ \int_\mathbb{R}\left(C(t,Se^y)-C(t,S)-S(e^y-1)\der{C}{S}(t,S)\right)k(y)dy=0
\end{multline}

su $[0,T]\times[0,+\infty]$ con opportune condizioni al bordo e condizione finale $C(T,S)=g(S)$ payoff dell'opzione. $k$ è un nucleo con una forte massa nell'intorno dello zero e code esponenziali.
In due dimensioni, supponendo l'indipendenza delle componenti di salto dei due sottostanti, tale equazione diventa:


\begin{multline}
 \der{C}{t}+\frac{\sigma_1^2}{2}S_1^2\dder{C}{S_1}+\frac{\sigma_2^2}{2}S_2^2\dder{C}{S_2}+\rho\sigma_1\sigma_2 S_1 S_2 \dmix{C}{S_1}{S_2}+
 r\der{C}{S_1}+r\der{C}{S_2}-rC+ \\
 + \int_\mathbb{R}\left(C(t,S_1e^{y},S_2)-C(t,S_1,S_2)-S_1(e^y-1)\der{C}{S}(t,S_1,S_2)\right)k_1(y)dy\\
 + \int_\mathbb{R}\left(C(t,S_1,S_2e^{y})-C(t,S_1,S_2)-S_2(e^y-1)\der{C}{S}(t,S_1,S_2)\right)k_2(y)dy=0 
\end{multline}
su $[0,T]\times[0,+\infty]^2$ con opportune B.C. e valore finale.
\\
Al variare delle condizioni finali e delle condizioni al contorno si possono descrivere altri tipi di opzioni. Un esempio interessante è il caso dell'opzione asiatica, che dipende dalla media del sottostante nel tempo. Infatti se consideriamo la media come seconda variabile, si ottiene un equazione simile a quella precedente.

\section{Strumenti}

\subsection{La libreria deal II}

Per realizzare questo progetto, l'idea è la libreria deal II per gli elementi finiti. Tale libreria permette un approccio pulito alle diverse parti necessarie alla costruzione di un programma ad elementi finiti. Sono presenti classi per le griglie, le matrici, le funzioni di base, la quadratura, i solutori e quant'altro necessario alla soluzione di un problema EF classico. Il punto principale sarà di aggiungere il termine integrale.

\subsection{Altri strumenti}

Abbiamo per ora incluso la libreria gsl per l'interpolazione del valore soluzione in punti non appartententi alla mesh. Siccome è una parte piuttosto pesante, si cercano modi di evitare l'interpolazione (vedere in seguito). La libreria deal II utilizza CMake di default per generare i makefile, e quindi abbiamo adottato tale metodo, adattandolo alle nostre necessità.

\section{Cosa è stato fatto}

In questa prima parte del progetto, abbiamo iniziato a fare i primi esperimenti con la libreria deal II. In particolare abbiamo:

\begin{itemize}
 \item Costruito un programma che risolve l'equazione PDE  (senza parte integrale) in una dimensione in un caso semplice utilizzando unicamente gli strumenti forniti dalla libreria.
 \item Abbiamo risolto lo stesso problema nel caso bi-dimensionale. Sebbene il comportamento qualitativo della soluzione è quello aspettato, i valori esatti non sono ancora giusti (confrontati con dei risultati dati da tool che risolvono l'equazione con metodi alle differenze finite).
 \item Abbiamo provato a risolvere l'equazione Integrodifferenziale in una dimensione in diversi modi (per il dettaglio vedere sotto). In due casi siamo riusciti ad ottenere un risultato corretto, ma non pienamente soddisfacenti.
\end{itemize}

\subsection{Metodologia}

La PDE (e la PIDE) in questione è trasformabile in un equazione a coefficienti costanti con la trasformazione $x=\ln{S} \; (\text{ o } x=\ln{\sfrac{S}{S_0}})$ e $C(t,S)=u(t,x)$. In tal caso diventa (1D):
{
\small
\begin{equation}
 \der{u}{t}+\frac{\sigma^2}{2}\dder{u}{x}+\left(r-\frac{\sigma^2}{2}\right)\der{u}{x}-ru+\int_\mathbb{R}\left( u(t,x+y)-u(t,x)-(e^y-1)\der{u}{x}\right)k(y)dy=0
\label{eq:costcoeff}
 \end{equation}
\normalsize
}
In molti casi, è possibile separare i tre pezzi dell'integrale, e trattare gli ultimi due addendi separatamente. Definendo 

\begin{equation*}
 \hat{\lambda}=\int_\mathbb{R}u(t,x)k(y)dy \qquad \hat{\alpha}=\int_\mathbb{R}(e^y-1)\der{u}{x}k(y)dy
\end{equation*}
L'equazione \eqref{eq:costcoeff} diventa: 

{

\begin{equation}
 \der{u}{t}+\frac{\sigma^2}{2}\dder{u}{x}+\left(r-\frac{\sigma^2}{2}-\hat{\alpha}\right)\der{u}{x}-r\hat{\lambda}u+\int_\mathbb{R}u(t,x+y)k(y)dy=0
\end{equation}
}

Per la parte temporale abbiamo utilizzato una discretizzazione differenze finite con schema di eulero implicito o Crank Nicolson, eccetto per la parte integrale che viene trattata esplicitamente e messa nel \emph{rhs}. Tale schema è stabile purché $\Delta t < \sfrac{1}{\hat{\lambda}}$. Gli elementi finiti scelti sono Q1, polinomi lineari sui quadrati, offerti gentilmente dalla libreria deal II. Il dominio su $S$ viene troncato all'intervallo $(S_{min},S_{max})$, opportunamente scelto. 
\\
La difficoltà si riduce dunque a integrare $u(t,x+y)k(y)$.
\\
Senza entrare nei dettagli, otteniamo una discretizzazione del tipo :

\begin{equation}
 M_1u^k=M_2u^{k+1}+J^{k+1}  \qquad \text{ per } k=M\dots1 \qquad \text{ e } \qquad u^{M}(S)=g(S) 
\end{equation}

Dove $M_1$ è la somma delle matrici date dagli elementi finiti (stiffnes, etc, etc) e $M_2$ è la matrice di massa divisa per il passo temporale. Il termine esplicito $J$ può essere calcolato in principio in diversi modi.

\subsection{La parte integrale $J$}

$k(y)$ è un nucleo che decresce rapidamente, è quindi possibile troncare il dominio d'integrazione ottenendo una soluzione di poco diversa (esistono stime a riguardo, qua non citate). L'integrale è allora da fare sull'intervallo $(B_l,B_u)$, con $B_l$ e $B_u$ opportunamente scelti. Si presentano due problemi con questo termine:

\begin{itemize}
 \item Nel caso generale $(S_{min},S_{max}) \subset (B_l,B_u)$, quindi il termine integrale oltre a essere non locale necessita dei valori di $u$ fuori dal dominio. Si sceglie di estendere il valore di $u$ utilizzando la condizione al bordo, pratica comune in questi casi.
 \item Il termine $u(t,x+y)$ non è facilmente trattabile in quanto il fatto di sommare $x+y$ introduce un possibile shift al di fuori dai nodi di griglia al quale bisogna stare attenti
\end{itemize}

\subsubsection{Possibili approcci}

Una prima differenza negli approcci all'ultimo termine integrale è la scelta di scrivere l'incognita $u$ subito come appartenente allo spazio $V_h$ dove sono ambientati gli elementi finiti, o prima integrare ottenendo una funzione $J(x)$ e scrivere quest'ultima come oggetto di $V_h$. Partiamo dal secondo approccio.

\subsubsection{Integrazione prima di discretizzazione}

Questa è una tecnica a quanto pare più diffusa (poiché utilizzato nelle differenze finite)
In questo caso, si tratta di calcolare: 

\begin{equation*}
 J_i=J(x_i)=\int_{B_l}^{B_u}u(t,x_i+y)k(y)dy
\end{equation*}
nei diversi nodi $x_i$ della griglia. In seguito si potra scrivere dunque 

\begin{equation*}
 \sum_{j=1}^N J_j\int_{x_{min}}^{x_{max}} \phi_i(x)\phi_j(x)dx
\end{equation*}

scrivibile come $M\underline{J}$, con $M$ matrice di massa, e aggiungere tale termine all'\emph{rhs}.

L'eq da risolvere a ogni passo temporale è dunque:

\begin{equation*}
 M_1u^k=M_2u^{k+1}+M\underline{J}^{k+1}  \qquad \text{ per } k=M\dots1
\end{equation*}


% \begin{equation*}
% \end{equation}
In tal caso ci sono due modi di calcolare il vettore $\underline{J}:$

\begin{itemize}
 \item Utilizzare una griglia qualunque d'integrazione, interpolando il valore di $u$ in $x_i+y$ se $x_i+y$ cade all'interno del dominio, o imponendo le condizioni al bordo se fuori dal dominio. Tale operazione, abbastanza semplice, è molto lenta. Per l'interpolazione abbiamo usato le funzioni della \emph{gsl}, utilizzando un metodo a spline.
 \item Allineare le griglie d'integrazione e del dominio in modo che $x_i+y$ coincida sempre con un nodo quando dentro il dominio. Questa procedura è semplice in 1D se il passo della griglia è costante. Questa procedura è più veloce, ma suppone che i nodi siano ordinati. 
\end{itemize}

In questi due casi abbiamo ottenuto risultati corretti in 1D, ma stiamo cercando di capire come si può fare in più dimensioni.

Una strada che abbiamo provato in questo caso è di usare il cambio di variabili $x+y=z$, e il termine da integrare diventa

\begin{equation*}
 J_i=J(x_i)=\int_{B_l}^{B_u}u(t,x_i+y)k(y)dy=\int_{B_l+x_i}^{B_u+x_i}u(t,z)k(z-x_i)dz
\end{equation*}

La griglia d'integrazione cambia passo a passo, ma anche in questo caso deve essere allineata con la griglia del dominio ed è necessario sfruttare l'ordinamento.

\subsubsection{Discretizzazione e in seguito integrazione}

Un altra strada più naturale è di scrivere prima la funzione $u$ come appartente allo spazio $V_h$. In tal caso si ha al passo k:



\begin{align*}
 & \int_{x_{min}}^{x_{max}}\phi_i(x)\left( \int_{B_l}^{B_u} \sum_{j=1}^N u_j^k \phi_j(x+y)k(y)dy \right)dx=\\ 
 &\sum_{j=1}^N  \int_{x_{min}}^{x_{max}}\int_{B_l}^{B_u} \phi_i(x)\phi_j(x+y)k(y)dxdy
\end{align*}

che si potrebbe anche scrivere, tramite un cambio di variabile prima della moltiplicazione per $\phi_i$ come

\begin{equation*}
 \sum_{j=1}^N  \int_{x_{min}}^{x_{max}}\int_{B_l+x}^{B_u+x} \phi_i(x)\phi_j(z)k(z-x)dzdx
\end{equation*}

In tal caso, si potrebbe scrivere come una matrice densa $A$ e l'equazione da risolvere a ogni passo temporale sarebbe:

\begin{equation*}
 M_1u^k=M_2u^{k+1}+Au^{k+1}  \qquad \text{ per } k=M\dots1
\end{equation*}

Sebbene si abbia una moltiplicazione matrice densa per vettore a ogni passo, la matrice va costruita un unica volta.\\
Per ora non siamo riusciti a ottenere risultati corretti con questa tecnica, ne con cambio di variabili, ne con quadrature ne cercando di calcolare esattamente. Questo approccio sembra prestarsi a una costruzione della matrice facendo un ciclo sui quadrati, che sarebbe estensibile più facilmente al 2D.

\subsubsection{Conclusioni parziali e futuri passi}

Riassumendo, dei due approcci per calcolare l'integrale, il primo ha dato riultati corretti ma sembra più difficile da estendere a due dimensioni in modo efficente. Il secondo fin'ora non ha dato risultati, probabilmente per errore negli algoritmi.\\ \\
Un terzo approccio da esplorare è il cambio di variabili $Se^y=z$ a livello equazione. In tal caso si ottiene un equazione a coefficienti non costanti, ma la parte integrale potrebbe essere più semplice.
\\ \\
L'altro grande filone su cui concentrarsi è l'estensione in modo che sia dimension-independent.
\end{document}
